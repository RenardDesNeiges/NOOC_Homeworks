\documentclass[11pt]{article}

\usepackage{sectsty}
\usepackage{graphicx}
\usepackage[T1]{fontenc}
\usepackage{epigraph} %quotes
\usepackage{amssymb} %math symbols
\usepackage{mathtools} %more math stuff
\usepackage{amsthm} %theorems, proofs and lemmas
\usepackage[ruled,vlined]{algorithm2e} %algoritms/pseudocode
\usepackage{bm}

\newcounter{thecnt}[section]
\def\thecnt{\arabic{}}

%% Theorem notation
\newtheorem{theorem}{Theorem}
\newtheorem{corollary}{Corollary}
\newtheorem{claim}{Claim}
\newtheorem{lemma}{Lemma}
\newtheorem{problem}{Problem}
\newtheorem{definition}{Definition}
\newtheorem{remark}{Remark}

%% declaring abs so that it works nicely
\DeclarePairedDelimiter\abs{\lvert}{\rvert}%
\DeclarePairedDelimiter\norm{\lVert}{\rVert}%

% Swap the definition of \abs* and \norm*, so that \abs
% and \norm resizes the size of the brackets, and the 
% starred version does not.
\makeatletter
\let\oldabs\abs
\def\abs{\@ifstar{\oldabs}{\oldabs*}}
%
\let\oldnorm\norm
\def\norm{\@ifstar{\oldnorm}{\oldnorm*}}
\makeatother

% Marges
\topmargin=-0.45in
\evensidemargin=0in
\oddsidemargin=0in
\textwidth=5.5in
\textheight=9.0in
\headsep=0.5in


\title{Homework Set 2 - Networks out of Control}
\date{\today}
\author{Titouan Renard}

\begin{document}
\maketitle	

\section*{Exercise 1}


\begin{claim}
    Given that $n$ is even, using Stirling's formula, we claim that:
    \[ (n-1)!! \approx \alpha n^{n/2}e^{-n/2}, \]
    for some $\alpha \in \mathbb{R},~\alpha>0$ to be determined.
\end{claim}

\begin{proof}
    Recall that the double factorial of a number $n \in \mathbb{Z}$, denoted $n!!$, is given by the expression: 
    \begin{align*}
        n!! = \prod_{k=0}^{\frac{n}{2}} (n-2k) = n \cdot (n-2)  \cdot ... 3 \cdot 1 && \text{for an odd number,} \\
        n!! = \prod_{k=0}^{\frac{n+1}{2}} (n-2k) = n \cdot (n-2)  \cdot ... 4 \cdot 2 && \text{for an even number.} 
    \end{align*}
    Since $n$ is even we have that $n-1$ is odd. Observe that the expression
    \[ (n-1)!! = (n-1) \cdot (n-3)  \cdot ... 3 \cdot 1 \]
    can be expressed as (let $n=2k$, by $n$ even $k\in\mathbb{Z}$): 
    \begin{align*}
        (n-1)!! = \frac{(n-1) \cdot (n-2) \cdot (n-3) \cdot ... 3 \cdot 2 \cdot 1}{(n-2) \cdot (n-4) \cdot ... 4 \cdot 2} \\
        = \frac{(2k-1) \cdot (2k-2) \cdot (2k-3) \cdot ... 3 \cdot 2 \cdot 1}{(2k-2) \cdot (2k-4) \cdot ... 4 \cdot 2} = \frac{(2k-1)!}{2(k-1)!}.
    \end{align*}
    Which we reduce into:
    \begin{align*}
        (n-1)!! =  \frac{(2k-1)!}{2(k-1)!} = \frac{\frac{1}{n}*(n)!}{2(n/2)!*\frac{2}{n}} = \frac{1}{4} \frac{(n)!}{(n/2)!},
    \end{align*}
    by Stirling's formula we further get:
    \begin{align*}
        \frac{(n)!}{4(n/2)!} \approx  \frac{n^{n} e^{-n} \sqrt{2 \pi n}}{4(n^{n/2} e^{-n/2} \sqrt{\pi n})} = \frac{1}{2\sqrt{2}}n^{n/2}e^{-n/2},
    \end{align*}
    where $1/2\sqrt{2} = \alpha$.
\end{proof}

\newpage

\section*{Exercise 2} 

\begin{definition}
    A graph $G=(V,E)$ is said to be $k$-connected if there are at least $k$ vertex disjoint path between any two vertices $u,v\in V$ in $G$.
\end{definition}

\begin{theorem}
    (Connectivity of $G(n,r)$). For $r\geq 3$. $G(,r)$ is $r$-connected a.a.s. .
\end{theorem}

\begin{definition}
    Partition the vertex set $V$ of the graph the graph $G=(V,E)$ into 3 $A,B,S$ disjoint partitions s.t. $A\cup S \cup B=V$. We say that the set $S\subset V$ \textbf{separates} $G$ if $\not \exists ~ (u,v) \in E$ s.t. $u\in A,~v\in B$. 
\end{definition}

\begin{remark}
    If a graph $G=(V,E)$ is $k$ connected $\iff$ the size of the smallest set $S$ that separates $A$ and $B$ is $k$.
\end{remark}

\begin{proof} (of theorem 1)
    We separate the proof into two subcases, pick an arbitrary large number $a_0 \in \mathbb{Z}^+$, we distinguish the proof between two components, the \textbf{small component case} for $a=|A|<a_0$ and the \textbf{large component case} $a=|A|>a_0$.
    \linebreak

    For the \textbf{small component case} the proof is included in the lecture notes.
    \linebreak

    For the \textbf{large component case} we use a proof similar to the case $2<a<a_0$ from the small component proof. Let $T\subseteq S$ be the subset of vertices in $S$ adjacent to $A$, let $t=|T|$ and $s=|S|$. To show our result, lower-bound $t$ and therefore $s$.
\end{proof}

\newpage

\section*{Exercise 3}


\end{document}
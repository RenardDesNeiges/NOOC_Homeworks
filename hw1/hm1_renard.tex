\documentclass[11pt]{article}

\usepackage{sectsty}
\usepackage{graphicx}
\usepackage[T1]{fontenc}
\usepackage{epigraph} %quotes
\usepackage{amssymb} %math symbols
\usepackage{mathtools} %more math stuff
\usepackage{amsthm} %theorems, proofs and lemmas
\usepackage[ruled,vlined]{algorithm2e} %algoritms/pseudocode
\usepackage{bm}

%% Theorem notation
\newtheorem{theorem}{Theorem}[section]
\newtheorem{corollary}{Corollary}[theorem]
\newtheorem{claim}{Claim}[section]
\newtheorem{lemma}[theorem]{Lemma}
\newtheorem{problem}{Problem}[section]

%% declaring abs so that it works nicely
\DeclarePairedDelimiter\abs{\lvert}{\rvert}%
\DeclarePairedDelimiter\norm{\lVert}{\rVert}%

% Swap the definition of \abs* and \norm*, so that \abs
% and \norm resizes the size of the brackets, and the 
% starred version does not.
\makeatletter
\let\oldabs\abs
\def\abs{\@ifstar{\oldabs}{\oldabs*}}
%
\let\oldnorm\norm
\def\norm{\@ifstar{\oldnorm}{\oldnorm*}}
\makeatother

% Marges
\topmargin=-0.45in
\evensidemargin=0in
\oddsidemargin=0in
\textwidth=5.5in
\textheight=9.0in
\headsep=0.5in


\title{Homework Set 1 - Networks out of Control}
\date{\today}
\author{Titouan Renard}

\begin{document}
\maketitle	

\section*{Exercise 1}


\begin{claim}
    Given $G(n,p)$ an Erdòs-Rény Graph. We claim that if $p\geq \frac{\log n}{\sqrt{n}}$ then, $\text{diag}(G) \leq 2$.
\end{claim}

\begin{proof}
    The proof starts in a similar way to \textbf{theorem 3.4} from the lecture notes. Let $X$ denote the number of vertices $u,v$ with no common neighbor $w$. Observe that if  $X \rightarrow 0$ a.a.s , then the claim is true (as in \textbf{theorem 3.4}). We will thus proceed by showing that $p\geq \frac{\log n}{\sqrt{n}}$ implies $X \rightarrow 0$. \\

    First, observe that for a given pair of vertices $u,v$ and some other vertex $w$, the probability that $w$ is a common neighbor of $u,v$ is given by $p^2$ (by definition of the Erdòs-Rény model). The probability that $w$ is \textit{not} a common neighbor is thus given by $1-p^2$. Observe that since there are $n-2$ vertices in $G$ (excluding $u,v$) the probability that $u,v$ have no common neighbor is given by $(1-p^2)^{n-1}$. \\
    
    Let $X_{u,v}$ be the indicator variable that takes value $1$ if $u,v,$ have no common neighbor and $0$ else, we have that $\mathbb{E}[X_{u,v}] = \mathbb{P}[u,v \text{ have no common neighbor}] = (1-p^2)^{n-1}$. Let $X$ be the indicator variable of the number of $u,v$ pairs that have no common neighbors, we have $X = \sum_{u,v \in V} X_{u,v}$. By the linearity of expectation we get that $\mathbb{E}[X] = \mathbb{E}[\sum_{u,v \in V} X_{u,v}] = \sum_{u,v \in V} \mathbb{E}[X_{u,v}]$. Using the first moment method (\textbf{theorem 3.2}) we argue that showing that $\mathbb{E}[X] \rightarrow 0$ is sufficient to show that $X\rightarrow 0$ a.a.s .

    \begin{align*}
        \mathbb{E}[X] = \sum_{u,v \in V} \mathbb{E}[X_{u,v}] = \sum_{u,v \in V} (1-p^2)^{n-2} = \binom{n}{2} \cdot (1-p^2)^{n-2}\\
    \end{align*}

    Which is where we start making wild upper bound guesses. 

    \begin{align*}
        \binom{n}{2} \cdot (1-p^2)^{n-2} \leq c \cdot n^2 \cdot (1-p^2)^{n} \\
        \leq c \cdot n^2 \cdot e^{-p^2 \cdot n}.
    \end{align*}

    We now need to show that $c \cdot n^2 \cdot e^{-p^2 \cdot n} \rightarrow 0$, to do so we take the log on both sides and argue that showing $\log \left(c \cdot n^2 \cdot e^{-p^2 \cdot n} \right) \rightarrow -\infty$ is sufficient. 

    \begin{align*}
        \log \left(c \cdot n^2 \cdot e^{-p^2\cdot n} \right) \rightarrow -\infty \\
        2 \log (n) - n \cdot p^2 \rightarrow -\infty \\
    \end{align*}

    Which is true if for some $c\in \mathbb{R}_+$ we have $p \geq \frac{\log n}{\sqrt{n}}$.
\end{proof}

\newpage

\section*{Exercise 2} Consider two families of random graphs $G_n = G(n,p)$ with $p = n^\frac{-6}{5}$ and independent $H_n = G(\log n, (\log n)^\frac{-8}{7})$. 

\begin{claim}
$H_n$ does not appear in $G_n$ for large $n$. 
\end{claim}

\begin{proof}
    First, we look at $H_n$ and identify a graph that is likely to appear in it, for readability we let $m  = (\log n)$ and consider:
    \[ H_m = G(m,m^\frac{-8}{7}) = H_n = G(\log n, (\log n)^\frac{-8}{7}).\]

    Consider a tree $T = (V_T,E_T)$ with $|V_T|=7$ and $|E_T|= 6$. Observe that by that \textbf{theorem 3.6} from the lecture notes the function $t_H(m) = m^\frac{-7}{6}$ is a threshold function for the apparition $T$ in $H$. And that we have: 
    \[ \frac{p_H(m)}{t(m) } = \frac{m^\frac{-8}{7}}{m^\frac{-7}{6}} \rightarrow + \infty. \]

    So $T$ is a subgraph of $H$ a.a.s. Now using that result observe that since $G_n(n,n^\frac{-6}{5})$ we can use that same threshold function to show that since 

    \[ \frac{p_G(n)}{t(n) } = \frac{n^\frac{-6}{5}}{m^\frac{-7}{6}} \rightarrow 0, \]

    $T$ is not a subgraph of $G_n$ a.a.s., now just observe that since $T \not \subset G$ and $T \subset H$ we have $T \subset H \not \subset G$.

\end{proof}

\newpage

\section*{Exercise 3}

Fix $k\in \mathbb{Z}_+$ \textbf{theorem 3.6} implies that $t_c(n)$ provides a threshold for the appearance of some cycles (i.e. if $k$-cycles appear in $G$ then $G$ contains some cycles and ${p(n)}/{t_c(n)} \rightarrow \infty$), but we still need to show that ${p(n)}/{t_c(n)} \rightarrow 0 ~ \forall ~k \in \mathbb{Z}_+ ~, k > 3$ in order to prove the claim from the datum.

\begin{claim}
    $t_c(n) = \frac{1}{n}$ is such that if $\frac{p(n)}{t_c(n)} = np(n) \rightarrow 0$ then the graph $G$ has no cycle.
\end{claim}

\begin{proof}
    We will prove our claim by computing the expected number of cycles (indicated by the random variable $X_n$) for any $n$. Observe that:

    \[ \mathbb{P}(\text{$G$ has a cycle}) \leq \mathbb{E}[X_n].\]

    Now observe that we have:
    \begin{align*}
        \mathbb{E}[X_n] = \sum_{S \in T} \mathbb{E}[\bm{1}_{C_s}] = \sum_{k \geq 3} \sum_{S \in T_k}  \mathbb{P}(C_s)
    \end{align*}

    Where $T$ denotes the set of all subsets of the vertices $V$ of $G$ that are distinct in the sense that a unique cycle could occur (we don't double-count cycles), $T_k$ the set of all subsets of size $k$ of all the vertices of $V$ (in the same sense), $C_s$ the event that the subset $S$ is a cycle. Since we work with an Erdòs-Rény model we have that:

    \begin{align*}
        \mathbb{P}(C_s) = p^{|S|}.
    \end{align*}

    Now let's look at $|T_k|$, since sets are ordered we have $|T_k|=\binom{n}{k}k!\cdot\frac{1}{2k} = \frac{(k-1)!}{2}\binom{n}{k}$. \textit{There are $\binom{n}{k}$ sets of $k$ different vertices we can pick out of $n$, we need them to be ordered so we multiply that by $k!$, but we don't care about all ordering, cycle direction and starting vertex are of no importance se we divide by $2k$.} We thus have:
    
    \begin{align*}
        \sum_{S \in T_k} \mathbb{P}(C_s) = \frac{(k-1)!}{2}\binom{n}{k}p^k.
    \end{align*}

    Which we can use to get an expression for the expected number of cycles:

    \begin{align*}
        \mathbb{E}[X_n] =  \sum_{k \geq 3} \frac{(k-1)!}{2}\binom{n}{k}p^k.
    \end{align*}

    Which we now bound using that $\binom{n}{k}k! \leq n^k$:

    \begin{align*}
        \mathbb{E}[X_n] =  \sum_{k \geq 3} \frac{(k-1)!}{2}\binom{n}{k}p^k 
        \leq  \sum_{k \geq 3} k! \binom{n}{k}p^k \leq  \sum_{k \geq 3} n^k p^k.
    \end{align*}

    Which allows us to bound with a geometric series:

    \begin{align*}
        \mathbb{E}[X_n] \leq  \sum_{k \geq 3} (n p)^k = \sum_{k \geq 0} (n p)^k - \sum_{k \geq 0}^2 (n p)^k \\
        = \frac{1}{1 - np} - \frac{1 - (np)^3}{1-np} = \frac{(np)^3}{1-np}.
    \end{align*}

    Now observe that ${p(n)}/{t_c(n)} \rightarrow 0$ is equivalent to $np(n) \rightarrow 0$, so $\mathbb{E}[X_n] \rightarrow 0$ which completes the proof.

\end{proof}

\end{document}
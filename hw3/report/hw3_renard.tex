\documentclass[11pt]{article}

\usepackage{sectsty}
\usepackage{graphicx}
\usepackage[T1]{fontenc}
\usepackage{epigraph} %quotes
\usepackage{amssymb} %math symbols
\usepackage{mathtools} %more math stuff
\usepackage{amsthm} %theorems, proofs and lemmas
\usepackage[ruled,vlined]{algorithm2e} %algoritms/pseudocode
\usepackage{bm}

\newcounter{thecnt}[section]
\def\thecnt{\arabic{}}

%% Theorem notation
\newtheorem{theorem}{Theorem}
\newtheorem{corollary}{Corollary}
\newtheorem{claim}{Claim}
\newtheorem{lemma}{Lemma}
\newtheorem{problem}{Problem}
\newtheorem{definition}{Definition}
\newtheorem{remark}{Remark}

%% declaring abs so that it works nicely
\DeclarePairedDelimiter\abs{\lvert}{\rvert}%
\DeclarePairedDelimiter\norm{\lVert}{\rVert}%

% Swap the definition of \abs* and \norm*, so that \abs
% and \norm resizes the size of the brackets, and the 
% starred version does not.
\makeatletter
\let\oldabs\abs
\def\abs{\@ifstar{\oldabs}{\oldabs*}}
%
\let\oldnorm\norm
\def\norm{\@ifstar{\oldnorm}{\oldnorm*}}
\makeatother

% Marges
\topmargin=-0.45in
\evensidemargin=0in
\oddsidemargin=0in
\textwidth=5.5in
\textheight=9.0in
\headsep=0.5in


\title{Homework Set 3 - Networks out of Control}
\date{\today}
\author{Titouan Renard}

\begin{document}
\maketitle	

\section*{Exercise 1}
\subsection*{Question 1}


\begin{claim}
    Consider two increasing events $A_1$ and $A_2$ having equal probabilities. Show that: 
    \begin{align}
        \mathbb{P}_p(A_1) \geq 1 - \sqrt{1 - \mathbb{P}_p(A_1 \cup A_2)} . \label{claim1}
    \end{align}
\end{claim}

\begin{proof}
    Start from:
    \begin{align*}
        1 - \mathbb{P}_p(A_1 \cup A_2) = \mathbb{P}_p(\overline{A_1} \cap \overline{A_2}),
    \end{align*}
    now observing that if $A_i$ is increasing then $\overline{A_i}$ is decreasing, we have, by then FKG inequality (which holds for both increasing and decreasing events):
    
    \begin{align*}
        \mathbb{P}_p(\overline{A_1} \cap \overline{A_2}) \geq \mathbb{P}_p(\overline{A_1}) \mathbb{P}_p(\overline{A_2}) = \mathbb{P}_p(\overline{A_1})^2, \\
        1 - \mathbb{P}_p(A_1 \cup A_2) = \mathbb{P}_p(\overline{A_1} \cap \overline{A_2}) \geq \mathbb{P}_p(\overline{A_1})^2, \\
        \sqrt{1 - \mathbb{P}_p(A_1 \cup A_2)} \geq \mathbb{P}_p(\overline{A_1}). \\
        \mathbb{P}_p(A_1)  \geq 1 - \sqrt{1 - \mathbb{P}_p(A_1 \cup A_2)}. \\
    \end{align*}
\end{proof}

\subsection*{Question 2} 
\textit{Suppose $\mathbb{P}_p(A_1) = \mathbb{P}_p(A_2) > 0$ but $A_1$ increasing while $A_2$ is decreasing, is (\ref{claim1}) still true?}\\


Pick a trivial example, consider a $2$-vertices (denoted $v_1$ and $v_2$) lattice with $p=1/2$ that an edge appears. Let $A$ denote the event that $\exists~ v_1 \longleftrightarrow v_2$ and $B$ $\nexists~ v_1 \longleftrightarrow v_2$. Observe that although $\mathbb{P}_p(A) = \mathbb{P}_p(B)$, $A$ is increasing and $B$ is decreasing. Now compute both sides of the inequality (\ref{claim1}).

\begin{align*}
    \text{left side} && \mathbb{P}_p(A) = 1/2 \\
    \text{righta side} && 1 - \sqrt{1 - \mathbb{P}_p(A_1 \cup A_2)}  =  1 - \sqrt{1 - 1} = 1,
\end{align*}
observe that $1\geq 1/2$. This is a counter example.\\

\textit{Below is the begining of a proof I had to go through to get the example, it clearly not required for the question feel free to ignore it.} \\

One easy way to see why (\ref{claim1}) is true is is simply to observe that the FKG inequality stops holding which breaks the bound above. Our example is the following: we will get our example starting from the proof of \textbf{Theorem 8.2 (FKG Inequality)} from the lecture notes. Consider an increasing event $A$ and a decreasing event $B$. Let $X = \bm{1}_A$ and $X = \bm{1}_B$, thus $X$ is an increasing random variable and $Y$ a decreasing random variable. \\ %As in the case \textbf{Theorem 8.2} proof one can reformulate the FKG inequality as $\mathbb{E}[XY] \leq \mathbb{E}[X] \mathbb{E}[Y]$

Pick $n=1$ so that $X$ and $Y$ are only function of the state $\omega(e_1)$ of the edge $e_1$. Pick any two states $\omega_1, \omega_2 \in \{0,1\}$. Since $X$ is increasing and $Y$ is decreasing we have: 
\begin{align*}
    (X(\omega_1)-X(\omega_2))(Y(\omega_1)-Y(\omega_2)) \leq 0
\end{align*}

with equality if $\omega_1 = \omega_2$. As, if $X(\omega_1)\geq X(\omega_2)$, one must have $Y(\omega_1)\leq Y(\omega_2)$ by the increasing-decreasing property (or vice-versa). Therefore: 

\begin{align*}
    0 \geq \sum_{\omega_1=0}^1 \sum_{\omega_2=0}^1 \overbrace{(X(\omega_1)-X(\omega_2))(Y(\omega_1)-Y(\omega_2))}^{\leq 0}  \overbrace{\mathbb{P}_p(\omega(e_1) = \omega_1)}^{\geq 0} \overbrace{\mathbb{P}_p(\omega(e_1) = \omega_2)}^{\geq 0}\\
    = \sum_{\omega_1=0}^1 X(\omega_1)Y(\omega_1) \mathbb{P}_p(\omega(e_1) = \omega_1) +  \sum_{\omega_2=0}^1 X(\omega_2)Y(\omega_2) \mathbb{P}_p(\omega(e_1) = \omega_2) \\
    -  \sum_{\omega_1=0}^1 \sum_{\omega_2=0}^1 (X(\omega_1)+X(\omega_2))(Y(\omega_1)-Y(\omega_2))  \mathbb{P}_p(\omega(e_1) = \omega_1) \mathbb{P}_p(\omega(e_1) = \omega_2)\\
    = 2(\mathbb{E}_p[XY]-\mathbb{E}_p[X]\mathbb{E}_p[Y]).
\end{align*}

From which we get:
\begin{align*}
    0 \geq \mathbb{E}_p[XY]-\mathbb{E}_p[X]\mathbb{E}_p[Y]\\
    \mathbb{E}_p[X]\mathbb{E}_p[Y] \geq \mathbb{E}_p[XY],
\end{align*}
which gives a "reversed" FKG inequality for the $n=1$ case. \\

\end{document}